\documentclass[notes]{beamer}
\usepackage[utf8]{inputenc}
\usepackage{hyperref}

\usetheme{Luebeck}
\usecolortheme{beaver}
\beamertemplatenavigationsymbolsempty
\setbeamertemplate{footline}[frame number]

\title{SOFTENG 370 Tutorial 3}
\author{Timo van Veenendaal}
\date{18 August 2020}

\begin{document}

\maketitle

\begin{frame}{Hello again}
    \begin{itemize}
	\item Welcome back to remote university
	\item Tutorials are still on at the normal time! They are recorded too!
	\item Want to meet? Pick a time at \url{https://calendly.com/timov/se370-office-hours} and I'll meet you on Zoom!
	\item Questions during tutorial? Unmute yourself and ask or put them in the Zoom chat
	\item Questions outside tutorial? Contact me, any way is good but my email is tvan508@aucklanduni.ac.nz
    \end{itemize}
\end{frame}

\begin{frame}{Plan}
    \begin{itemize}
	\item Recap of \texttt{mmap}
	\item Condition variables
	\item Assignment Q\&A
    \end{itemize}
    Example files from tutorials are available on GitHub: \url{https://github.com/timovv/se370-tutorials/}.
\end{frame}

\begin{frame}[fragile]{Memory mapping}
     \begin{itemize}
	 \item \texttt{mmap} lets you create a \textit{memory object} (memory mapping) in the process's memory. It is a very powerful function with a lot of use cases.
	\item From the man page (\texttt{man mmap}):

\begin{verbatim}
The mmap() function shall establish a mapping
between an address space of a process and a
memory object.

The mmap() function shall be supported for
the following memory objects:
     *  Regular files
     *  Shared memory objects
     *  Typed memory objects
\end{verbatim}
     \end{itemize}
\end{frame}

\begin{frame}[fragile]{Memory mapping}
    \texttt{mmap} has the following signature:
\begin{verbatim}
void *mmap(void *addr, size_t len, int prot, int flags,
           int fildes, off_t off);
\end{verbatim}
    \begin{itemize}
	\item \texttt{void *addr}: address within the process's memory space where the memory mapping should be created. Pass \texttt{NULL} to let the OS to figure it out.
	\item \texttt{size\_t len}: size (in bytes) of the memory mapping.
	\item \texttt{int prot}: protection flags. Some combination of \texttt{PROT\_READ}, \texttt{PROT\_WRITE}, and \texttt{PROT\_EXEC} ORed together using \texttt{|}. Specifies what can be done with the mapped area.
    \end{itemize}
\end{frame}

\begin{frame}[fragile]{Memory mapping}
    \texttt{mmap} has the following signature:
\begin{verbatim}
void *mmap(void *addr, size_t len, int prot, int flags,
           int fildes, off_t off);
\end{verbatim}
    \begin{itemize}
	\item \texttt{int flags}: bitwise or of these flags: \texttt{MAP\_ANONYMOUS} (mapping not attached to a file), \texttt{MAP\_SHARED} (share mapping between processes) and \texttt{MAP\_PRIVATE} (unique copy of memory for each process).
	\item \texttt{int fildes}: a file descriptor for a file previously opened using \texttt{open()}. If no file, set to -1.
	\item \texttt{int off}: offset into the file for the memory mapping.
	\item Returns a \texttt{void *} pointing to the created memory mapping.
	\item Q: What arguments for \texttt{mmap} would create an area for processes can communicate with?
    \end{itemize}
\end{frame}

\begin{frame}{Condition variables}
    \begin{itemize}
	\item Assignment step 4
	\item Can be used to block a thread, or multiple threads at the same time, until another thread both modifies a shared variable (the \textit{condition}), and signals the condition variable.
	\item \texttt{pthread\_cond\_wait}: wait on the condition variable to be signalled by another thread.
	\item \texttt{pthread\_cond\_signal}: signal the condition variable, releasing one \textbf{or more} waiting threads. Called after changing some condition.
	\item Used in combination with a mutex which protects the condition.
	\item Demo: \texttt{condition\_vars.c}
    \end{itemize}
\end{frame}

\begin{frame}{Q\&A}
    \begin{itemize}
	\item Questions about the assignment?
    \end{itemize}
\end{frame}
\end{document}

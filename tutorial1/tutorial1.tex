\documentclass[notes]{beamer}
\usepackage[utf8]{inputenc}
\usepackage{hyperref}

\usetheme{Luebeck}
\usecolortheme{beaver}
\beamertemplatenavigationsymbolsempty
\setbeamertemplate{footline}[frame number]

\title{SOFTENG 370 Tutorial 1}
\author{Timo van Veenendaal}
\date{4 August 2020}

\begin{document}

\maketitle

\begin{frame}{Hello}
I am Timo, you might remember me from last year. I am the tutor for 370 this semester.
    \begin{itemize}
        \item In these tutorials I'll cover some things about the previous week's content
        \item There should be some time for Q\&A about assignments, past tests, etc. most of the time -- I'll hang around at the end of the tutorial to answer questions
        \item Alternatively, email me at tvan508@aucklanduni.ac.nz to set a time
        \item If you have questions, ask on Piazza if possible instead of emailing so that the rest of the class can see!
    \end{itemize}
\end{frame}
\begin{frame}{Plan}
    \begin{enumerate}
        \item Kernel and user modes
        \item Linux
        \item C on Linux
    \end{enumerate}
\end{frame}
\begin{frame}{Kernel}
    \begin{alertblock}{Question}
    What is kernel mode? How does it differ from user mode?
    \end{alertblock}
    \pause
    \begin{itemize}
        \item Having these two modes requires support from the hardware (i.e. the processor)
        \item When the processor is in kernel mode, all instructions can be executed, including  \textbf{privileged instructions} which cannot be executed in user mode.
        \item What are some examples of privileged instructions?
        \item Why are privileged instructions (and kernel mode) needed?
        \item What happens if a program attempts to execute a privileged instruction in user mode?
    \end{itemize}
\end{frame}
\begin{frame}{Kernel}
\begin{alertblock}{Question}
    How do ordinary programs get access to privileged instructions?
\end{alertblock}
\begin{itemize}
    \pause
    \item Through \textbf{system calls}
    \pause
    \item A system call is a hardware mechanism that causes the processor to jump to a predefined location and enter kernel mode.
    \item This location will contain some special code that is part of the kernel to handle the call. The kernel can of course use privileged instructions to complete the call
    \item The kernel can run checks to make sure that the caller has permission to do what they want to do
    \item When the kernel is finished handling the call, it tells the processor to return to user mode and then returns execution to the caller
\end{itemize}
\end{frame}
\begin{frame}{Linux}
    \begin{itemize}
        \item You need access to Linux (or similar) for the assignments
        \begin{itemize}
            \item You may also be able to get away with developing on macOS since it is also UNIX-based but you could run in to issues
        \end{itemize}
        \item A few options
        \begin{enumerate}
            \item Windows Subsystem for Linux
            \item Virtual machine
            \item FlexIT virtual machine
            \item Or dual boot...?
        \end{enumerate}
        \item Pick a distribution
        \begin{itemize}
            \item I recommend either Ubuntu or Manjaro if you haven't used Linux much before
            \item Otherwise use a distro of your choice
        \end{itemize}
        \item Hopefully you are all comfortable with \texttt{bash} from 206
    \end{itemize}
\end{frame}
\begin{frame}{Windows Subsystem for Linux (WSL)}
    \begin{itemize}
        \item The most convenient way to use Linux if your primary OS is Windows
        \item WSL 2, recently released, is essentially a wrapper around a Linux virtual machine. The older version of WSL is implemented differently
        \item Has excellent integration with VS Code through the ``Remote -- WSL'' extension (you'll want the C/C++ extension too)
        \item See \url{https://docs.microsoft.com/en-us/windows/wsl/install-win10} for installation instructions.
        \item Also check out the new Windows Terminal: \url{https://github.com/microsoft/terminal}
        \item Demo later
    \end{itemize}
\end{frame}
\begin{frame}{Virtual machine}
\begin{itemize}
    \item VirtualBox is free and good. A guide to installing Ubuntu on VirtualBox for Windows is available here: \url{https://brb.nci.nih.gov/seqtools/installUbuntu.html}
    \begin{itemize}
        \item You might have used this in 206 last year
    \end{itemize}
    \item Hyper-V is another alternative if using Windows Pro and is built in to the OS so you don't need to install anything
\end{itemize}
\end{frame}
\begin{frame}{FlexIT}
\begin{itemize}
    \item Ubuntu VM in the cloud
    \item Requires installation of VMWare Horizon client (see instructions)
    \item Instructions at \url{https://www.auckland.ac.nz/en/students/my-tools/flex-it/flexit-guide.html}
    \begin{itemize}
        \item Look for Ubuntu in the listing
    \end{itemize}
    \item Demo
\end{itemize}
\end{frame}
\begin{frame}{Demo}
Demo of FlexIT and WSL now
\end{frame}
\begin{frame}{The C Programming Language}
\begin{itemize}
    \item You might remember C from ENGGEN 131 (and maybe even COMPSYS 201)
    \item The way you will use C in this course is quite different to how it is approached in 131
    \begin{itemize}
        \item Feels more `low-level': heavy use of pointers, use of OS APIs (beyond just \texttt{printf})...
    \end{itemize}
    \item Compile on Linux using \texttt{gcc}
\end{itemize}
\end{frame}
\begin{frame}{Compiling and \texttt{gcc}}
\begin{itemize}
    \item You might need to install \texttt{gcc}
    \begin{itemize}
        \item Ubuntu: \texttt{sudo apt install build-essential}, you may have to do \texttt{sudo apt update} first
        \item Equivalents exist for other distributions, Google if unsure
    \end{itemize}
    \item To compile: \texttt{gcc -o <executable\_name> sourcefile.c}
    \item Then to execute: \texttt{./executable\_name}
    \item Let's do a demo
\end{itemize}
\end{frame}
\begin{frame}{Questions}
Any questions? About...
\begin{itemize}
    \item The course in general?
    \item Installing Linux?
    \item Using C and \texttt{gcc}?
\end{itemize}
Otherwise see you all next week
\end{frame}
\end{document}

\documentclass[notes]{beamer}
\usepackage[utf8]{inputenc}
\usepackage{hyperref}

\usetheme{Luebeck}
\usecolortheme{beaver}
\beamertemplatenavigationsymbolsempty
\setbeamertemplate{footline}[frame number]

\title{SOFTENG 370 Tutorial 6}
\author{Timo van Veenendaal}
\date{21 September 2020}

\begin{document}

\maketitle

\begin{frame}{Welcome back}
    Hope everyone had a good break! This week and next week's tutorial are remote on Zoom again. We'll see about the weeks after that.

    Check that the recording is on now.
\end{frame}
\begin{frame}{Plan}
    \begin{enumerate}
        \item Assignment 2
        \item Python
        \item Q\&A
    \end{enumerate}
\end{frame}
\begin{frame}{Assignment 2}
    \begin{itemize}
        \item Theme: file systems
        \item Based on a Python binding of \texttt{libfuse}, a library which allows for filesystems to be defined in user space (see lectures)
        \item 3 parts:
            \begin{itemize}
                \item Part 1: play around with the provided code and answer some questions
                \item Part 2: have a look at how the memory-based file system (\texttt{memory.py}) works and answer more questions (actually just 1 question)
                \item Part 3: make your own file system!
            \end{itemize}
    \end{itemize}
\end{frame}
\begin{frame}{Python syntax}
    A bit different to Java, but shouldn't be too hard to pick up:
    \begin{itemize}
        \item No semicolons
        \item Python has lists and dictionaries built in
        \item Classes can have multiple inheritance
        \item Spaces and stuff matter
        \item Let's take a look at some of the assignment code to explore some of the differences
    \end{itemize}
\end{frame}
\begin{frame}{\texttt{os} module}
    \begin{itemize}
        \item Python's \texttt{os} module (\texttt{import os}) has a few different low-level helpers which you will need in Part 3.
        \item \texttt{os.open}, \texttt{os.mkdir}, \texttt{os.mknod}, \texttt{os.path.join}, \dots
        \item Many of them map directly to Linux system calls (so you can check the man page: \texttt{man 2 mknod}, \texttt{man 2 open}, etc...)
        \item These functions (and more) may be needed to implement your file system.
        \item Unsure? Check out the docs: \url{https://docs.python.org/3/library/os.html}.
    \end{itemize}
\end{frame}
\begin{frame}{Questions?}
    \begin{itemize}
        \item Questions on the assignment or on course content from this week?
        \item Otherwise, see you next week. Next week will be more on A2.
    \end{itemize}
\end{frame}
\end{document}
